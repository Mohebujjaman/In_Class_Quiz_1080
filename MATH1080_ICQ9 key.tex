

\documentclass[paper=a4, fontsize=11pt]{scrartcl} % A4 paper and 11pt font size

\usepackage[T1]{fontenc} % Use 8-bit encoding that has 256 glyphs
\usepackage{fourier} % Use the Adobe Utopia font for the document - comment this line to return to the LaTeX default
\usepackage[english]{babel} % English language/hyphenation
\usepackage{amsmath,amsfonts,amsthm} % Math packages
\newcommand{\ds}{\displaystyle}
\usepackage{lipsum} % Used for inserting dummy 'Lorem ipsum' text into the template

\usepackage{sectsty} % Allows customizing section commands
\allsectionsfont{\centering \normalfont\scshape} % Make all sections centered, the default font and small caps

\usepackage{fancyhdr} % Custom headers and footers
\pagestyle{fancyplain} % Makes all pages in the document conform to the custom headers and footers
\fancyhead{} % No page header - if you want one, create it in the same way as the footers below
\fancyfoot[L]{} % Empty left footer
\fancyfoot[C]{} % Empty center footer
%\fancyfoot[R]{\thepage} % Page numbering for right footer
\renewcommand{\headrulewidth}{0pt} % Remove header underlines
\renewcommand{\footrulewidth}{0pt} % Remove footer underlines
\setlength{\headheight}{13.6pt} % Customize the height of the header

\numberwithin{equation}{section} % Number equations within sections (i.e. 1.1, 1.2, 2.1, 2.2 instead of 1, 2, 3, 4)
\numberwithin{figure}{section} % Number figures within sections (i.e. 1.1, 1.2, 2.1, 2.2 instead of 1, 2, 3, 4)
\numberwithin{table}{section} % Number tables within sections (i.e. 1.1, 1.2, 2.1, 2.2 instead of 1, 2, 3, 4)

\setlength\parindent{0pt} % Removes all indentation from paragraphs - comment this line for an assignment with lots of text
\usepackage[usenames, dvipsnames]{color}
\usepackage{lastpage}
\usepackage{fancyhdr}
\cfoot{\thepage\ of \pageref{LastPage}}
\usepackage{mathrsfs}
\usepackage{graphicx}

%----------------------------------------------------------------------------------------
%	TITLE SECTION
%----------------------------------------------------------------------------------------

\newcommand{\horrule}[1]{\rule{\linewidth}{#1}} % Create horizontal rule command with 1 argument of height

\title{	
\normalfont \normalsize 
\textsc{Jaman, Clemson University, MATH1080 - 1} \\ [25pt] % Your name, university, class
\horrule{0.5pt} \\[0.4cm] % Thin top horizontal rule
\huge In-class Quiz \#9 Key (Total Points 23) \\ % The assignment title
\horrule{2pt} \\[0.5cm] % Thick bottom horizontal rule
}

\author{Date:} % The due date

\date{\normalsize March 28, 2016} % A custom date

\begin{document}

\maketitle % Print the title

\begin{flushleft}
\begin{tabular}{l l}
Name: \rule{3.2in}{.01cm}  & {}%Table number: \rule{1in}{.01cm}\\
\end{tabular}
\end{flushleft}

%----------------------------------------------------------------------------------------
%	Directions
%----------------------------------------------------------------------------------------

\section*{\textbf{Directions:}}

No calculator or notes may be used.  Read each question very carefully.  In order to receive full credit, you must:
\begin{enumerate}
\item Show legible and logical (relevant) justification which supports your final answer.
\item Use complete and correct mathematical notation.
\item Include proper units, if necessary.
\item Give exact numerical values whenever possible.
\item Follow the directions given for the problem.
\end{enumerate}
\vspace{.1in}

\newpage

\begin{enumerate}
\item For the series $\ds\sum_{n=1}^{\infty}\frac{n^2\cos(n\pi)}{n^3+4}$
\begin{enumerate} 
\item (4 points)\hspace{4mm}Test for convergence or divergence.\\\\
\textbf{Solution:}\\ Here\hspace{2mm} $\ds\cos(n\pi)=(-1)^n$. Therefore the given series $\ds\sum_{n=1}^{\infty}\frac{n^2\cos(n\pi)}{n^3+4}=\ds\sum_{n=1}^{\infty}\frac{(-1)^nn^2}{n^3+4}$ is an alternating series, and $b_n=\ds\frac{n^2}{n^3+4}$. We want to apply Alternating series Test,\\\\
\textbf{Condition (1):} We want to show $\ds b_{n+1}\le b_n$\\
Let us consider a function $f(x)=\ds\frac{x^2}{x^3+4}$ where $x\ge 1$
$$f'(x)=\ds\frac{(x^3+4)2x-x^2(3x^2)}{(x^3+4)^2}=\ds\frac{x(8-x^3)}{(x^3+4)^2}$$
Observe that $f'(x)<0$  when $8-x^3<0\implies x^3>8\implies x>2$\\
That is $f(x)$ decreases when $x>2$, and hence $b_n$ decreases as $n>2$
$$b_{n+1}\le b_n\hspace{3mm}\text{for all}\hspace{2mm}n>2\hspace{2mm}\text{that is after first two terms.}\hspace{2mm}\textcolor{red}{(2\hspace{2mm} points)}$$

\textbf{Condition (2):}  $$\lim\limits_{n\rightarrow\infty}b_n=\ds\lim\limits_{n\rightarrow\infty}\ds\frac{n^2}{n^3+4}=\ds\lim\limits_{n\rightarrow\infty}\ds\frac{1}{n+\frac{4}{n^2}}=0$$
By Alternating series Test, the given series is convergent. \textcolor{red}{(2\hspace{2mm} points)}\\\\
\item (1 points)\hspace{4mm} If it diverges, say "Divergent". If it converges, suppose the 3\textit{rd} partial sum of the series is used to approximate the sum of the series. Find the best bound on the absolute value of the remainder (error) for this approximation.\\\\
\textbf{Solution:}$$|R_3|\le b_4$$
$b_4$ is the best bound on error and $b_4=\frac{4^2}{4^3+4}=\frac{16}{64}=\frac{4}{17}$  \textcolor{red}{(1\hspace{1mm} point)}
\end{enumerate}

\newpage
\item (4 points) It can be shown by Integral Test that the series $\ds\sum_{n=1}^{\infty}\frac{1}{n^4}$ converges. Using the Remainder Estimate for Integral Test, find the bound on the remainder when the sum of the first three terms of the series is used to approximate the sum of the series.\\\\
\textbf{Solution:}  In this case $\ds f(x)=\frac{1}{x^4}$
$$\ds\int_{4}^{\infty}f(x)\text{  }dx\le R_3\le\ds\int_{3}^{\infty}f(x)\text{  }dx\hspace{6mm}\textcolor{red}{(1\hspace{1mm} point )}$$
$\ds\int_{n}^{\infty}\frac{1}{x^4}\text{  }dx=\ds\lim\limits_{t\rightarrow\infty}\ds\int_{n}^{t}\frac{1}{x^4}\text{  }dx=\ds\lim\limits_{t\rightarrow\infty}\left[-\frac{3}{x^3}\right]_{n}^t=\ds\lim\limits_{t\rightarrow\infty}\left[-\frac{3}{t^3}+\frac{3}{n^3}\right]=\ds\frac{3}{n^3}\hspace{6mm}\textcolor{red}{(2\hspace{1mm} points )}$\\\\
Therefore,
$$\frac{3}{4^3}\le R_3\le\ds\frac{3}{3^3}\hspace{6mm}\textcolor{red}{(1\hspace{1mm} point )}$$
$$\frac{3}{64}\le R_3\le\ds\frac{1}{9}$$

\item (5 points) Determine whether the series $\ds \sum_{n=1}^{\infty}\ds\ln\frac{n+1}{n}$ converges or diverges?\\\\
\textbf{Solution: }  Using Logarithmic property we can write
$$\ds \sum_{n=1}^{\infty}\ds\ln\frac{n+1}{n}=\sum_{n=1}^{\infty}\ln(n+1)-\ln(n)$$
$S_1=\ln(2)-0$\\\\
$S_2=(\ln(2)-0)+(\ln(3)-\ln(2))$\\\\
$S_3=\big(\ln(2)-0\big)+\big(\ln(3)-\ln(2)\big)+\big(\ln(4)-\ln(3)\big)$\\\\
$S_4=\big(\ln(2)-0\big)+\big(\ln(3)-\ln(2)\big)+\big(\ln(4)\ln(3)\big)+\big(\ln(5)-\ln(4)\big)$\\\\
:\\\\
$S_n=\big(\ln(2)-0\big)+\big(\ln(3)-\ln(2)\big)+\big(\ln(4)-\ln(3)\big)+\cdots\cdots+\big(\ln(n)-\ln(n-1)\big)+\big(\ln(n+1)-\ln(n)\big)$\\\\
Telescoping sum implies
$$S_n=\ln(n+1)\hspace{6mm}\textcolor{red}{(3\hspace{2mm} points )}$$
$\lim\limits_{n\rightarrow\infty}S_n=\lim\limits_{n\rightarrow\infty}\ln(n+1)=\infty\hspace{6mm}\textcolor{red}{(1\hspace{2mm} point )}$\\\\
The sequence of partial sum diverges, therefore the series diverges.\hspace{6mm}\textcolor{red}{(1\hspace{2mm} point )}
\newpage

\item  Use the test of your choice to determine whether the following series are convergent or divergent.
\begin{enumerate}
\item (4 points) $\sum\limits_{n=1}^\infty (-1)^n \sin \left( \frac{\pi}{n} \right)$\\\\
\textbf{Solution:} It is an alternating series with $b_n=\sin \left( \frac{\pi}{n} \right)>0\hspace{3mm}n\ge 2$\\\\
Consider the function $f(x)=\sin \left( \frac{\pi}{x} \right)$, then $f'(x)=-\frac{\pi}{x^2}\cos \left( \frac{\pi}{x} \right)<0\hspace{3mm}\text{for all }\hspace{2mm}x>2$\\\\
That is, $f(x)$ decreases for all $x>2$, and thus $b_{n+1}\le b_n\hspace{3mm}\text{for all}\hspace{2mm}n>2$\hspace{2mm1} \textcolor{red}{(2\hspace{1mm}points)}\\\\
Again, $$\lim\limits_{n\rightarrow\infty}b_n=\lim\limits_{n\rightarrow\infty}\sin \left( \frac{\pi}{n} \right)=\sin \left(\lim\limits_{n\rightarrow\infty} \frac{\pi}{n} \right)=\sin (0)=0\hspace{2mm}\textcolor{red}{(1\hspace{1mm} point)}$$
By Alternating Series Test, the given series is convergent.\hspace{2mm}\textcolor{red}{(1\hspace{1mm} point)}\\\\
\item (5 points) $\ds\sum_{n=1}^{\infty}\frac{n!}{3^nn^2}$\\\\
\textbf{Solution: } Let $a_n=\frac{n!}{3^nn^2}\hspace{2mm}\textcolor{red}{(1\hspace{1mm} point)}$, then
$$\lim\limits_{n\rightarrow\infty}\left|\frac{a_{n+1}}{a_n}\right|\hspace{2mm}\textcolor{red}{(1\hspace{1mm} point)}=\ds\lim\limits_{n\rightarrow\infty}\left|\frac{(n+1)!}{3^{n+1}(n+1)^2}\frac{3^nn^2}{n!}\right|=\ds\lim\limits_{n\rightarrow\infty}\ds\frac{(n+1)n^2}{3(n+1)^2}=\ds\lim\limits_{n\rightarrow\infty}\ds\frac{n^3+n^2}{3(n^2+2n+1)}=\infty$$
\hspace{2mm}\textcolor{red}{(2\hspace{1mm} points)}\\
By Ratio Test, the given series, diverges.
\hspace{2mm}\textcolor{red}{(1\hspace{1mm} point)}
\end{enumerate}


\end{enumerate}

%----------------------------------------------------------------------------------------

\end{document}
