

\documentclass[paper=a4, fontsize=11pt]{scrartcl} % A4 paper and 11pt font size

\usepackage[T1]{fontenc} % Use 8-bit encoding that has 256 glyphs
\usepackage{fourier} % Use the Adobe Utopia font for the document - comment this line to return to the LaTeX default
\usepackage[english]{babel} % English language/hyphenation
\usepackage{amsmath,amsfonts,amsthm} % Math packages
\newcommand{\ds}{\displaystyle}
\usepackage{lipsum} % Used for inserting dummy 'Lorem ipsum' text into the template

\usepackage{sectsty} % Allows customizing section commands
\allsectionsfont{\centering \normalfont\scshape} % Make all sections centered, the default font and small caps

\usepackage{fancyhdr} % Custom headers and footers
\pagestyle{fancyplain} % Makes all pages in the document conform to the custom headers and footers
\fancyhead{} % No page header - if you want one, create it in the same way as the footers below
\fancyfoot[L]{} % Empty left footer
\fancyfoot[C]{} % Empty center footer
%\fancyfoot[R]{\thepage} % Page numbering for right footer
\renewcommand{\headrulewidth}{0pt} % Remove header underlines
\renewcommand{\footrulewidth}{0pt} % Remove footer underlines
\setlength{\headheight}{13.6pt} % Customize the height of the header

\numberwithin{equation}{section} % Number equations within sections (i.e. 1.1, 1.2, 2.1, 2.2 instead of 1, 2, 3, 4)
\numberwithin{figure}{section} % Number figures within sections (i.e. 1.1, 1.2, 2.1, 2.2 instead of 1, 2, 3, 4)
\numberwithin{table}{section} % Number tables within sections (i.e. 1.1, 1.2, 2.1, 2.2 instead of 1, 2, 3, 4)

\setlength\parindent{0pt} % Removes all indentation from paragraphs - comment this line for an assignment with lots of text

\usepackage{lastpage}
\usepackage{fancyhdr}
\cfoot{\thepage\ of \pageref{LastPage}}
\usepackage{mathrsfs}
\usepackage{graphicx}

%----------------------------------------------------------------------------------------
%	TITLE SECTION
%----------------------------------------------------------------------------------------

\newcommand{\horrule}[1]{\rule{\linewidth}{#1}} % Create horizontal rule command with 1 argument of height

\title{	
\normalfont \normalsize 
\textsc{Jaman, Clemson University, MATH1080 - 1} \\ [25pt] % Your name, university, class
\horrule{0.5pt} \\[0.4cm] % Thin top horizontal rule
\huge In-class Quiz \#8 Key \\ % The assignment title
\horrule{2pt} \\[0.5cm] % Thick bottom horizontal rule
}

\author{Date:} % The due date

\date{\normalsize March 21, 2016} % A custom date

\begin{document}

\maketitle % Print the title

\begin{flushleft}
\begin{tabular}{l l}
Name: \rule{3.2in}{.01cm}  & {}%Table number: \rule{1in}{.01cm}\\
\end{tabular}
\end{flushleft}

%----------------------------------------------------------------------------------------
%	Directions
%----------------------------------------------------------------------------------------

\section*{\textbf{Directions:}}

No calculator or notes may be used.  Read each question very carefully.  In order to receive full credit, you must:
\begin{enumerate}
\item Show legible and logical (relevant) justification which supports your final answer.
\item Use complete and correct mathematical notation.
\item Include proper units, if necessary.
\item Give exact numerical values whenever possible.
\item Follow the directions given for the problem.
\end{enumerate}
\vspace{.1in}

\newpage

\begin{enumerate}
\item For the series $\ds\sum_{n=2}^{\infty}\frac{1}{(n-1)^2}$
\begin{enumerate}
\item (2 point)\hspace{4mm}Show that the Integral Test may be applied to this series by verifying that the three conditions fo the  Integral Test hold for this series.\\\\
\textbf{Solution:} Let $\ds f(x)=\ds\frac{1}{(x-1)^2}$ Clearly $\ds f(x)$ is continuous and positive for all $\ds x\ds>\ds 1$.\\\\ Want to show (WTS): $\ds f(x)$ is decreasing.\\
Let's take its derivative:
$$f'(x)=\ds\frac{-2}{(x-1)^3}$$
Clearly, $f'(x)<0$ for all $\ds x>1$, which implies $f(x)$ decreases for all $x>1$ .
Therefore the conditions for Integral Test are satisfied.

\item (3 points)\hspace{4mm} Use Integral Test to determine if the series converges or diverges. Show all work analytically.\\\\
\textbf{Solution:}\\\\
$\ds\int_{2}^{\infty}\frac{1}{(x-1)^2}\ds\text{  } dx=\ds\lim\limits_{t\rightarrow\infty}\int_{2}^{t}\frac{1}{(x-1)^2}\text{  }dx$\\\\
Suppose $\ds u=x-1$ then $\ds du=dx$\\\\
$\ds\int_{2}^{\infty}\frac{1}{(x-1)^2}\ds\text{  } dx=\lim\limits_{t\rightarrow\infty}\int_{1}^{t-1}u^{-2}\text{  }du=\ds\lim\limits_{t\rightarrow\infty}\left[-\frac{1}{u}\right]_1^{t-1}=\lim\limits_{t\rightarrow\infty}\left[-\frac{1}{t-1}+1\right]=1$\\\\
The improper integral converges, therefore by Integral Test, the given series converges.
\end{enumerate}

\newpage
\item (5 points) Determine if the series $\ds\sum_{n=1}^{\infty}\frac{n+5}{\sqrt[3]{n^5+n^2}}$ converges or diverges.
\\\\
\textbf{Solution:}\\\\ Keeping only the leading order terms in numerator and denominator, we can write $b_n=\frac{n}{n^{5/3}}=\frac{1}{n^{2/3}}$. The series $\ds\sum_{n=1}^{\infty}b_n=\ds\sum_{n=1}^{\infty}\frac{1}{n^{2/3}}$ is a $p-$series with $p=\frac{2}{3}<1$, and thus diverges.\\\\
Now say $\ds a_n=\ds\frac{n+5}{\sqrt[3]{n^5+n^2}}$, then\hspace{2mm} $$\ds\frac{a_n}{b_n}=\ds\frac{n+5}{\sqrt[3]{n^5+n^2}}\cdot\frac{n^{2/3}}{1}=\ds\frac{n^{5/3}+5n^{2/3}}{\sqrt[3]{n^5+n^2}}$$
$\ds\lim\limits_{n\rightarrow\infty}\ds\frac{a_n}{b_n}=\lim\limits_{n\rightarrow\infty}\ds\frac{1+\frac{5}{n}}{{\sqrt[3]{1+\frac{1}{n^3}}}}=1$ which is $0<1<\infty$\\\\
Hence by Limit Comparison Test, the series
$\ds\sum_{n=1}^{\infty}\frac{n+5}{\sqrt[3]{n^5+n^2}}$ diverges



\end{enumerate}

%----------------------------------------------------------------------------------------

\end{document}
